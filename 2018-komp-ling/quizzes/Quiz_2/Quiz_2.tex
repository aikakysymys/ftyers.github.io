

Quiz 2


1. 
( "drawing_1" in the folder)

2. 
Examples: 1)"He's got", 2) "Mary's cat". To fix "He's got", I would compose something like:
 's:~is <= _ [epsilon]<v><3form>:
 To fix  "Mary's cat",  's:~is <= _ [[epsilon]<adj>* epsilon]<n>*:  <v><3form>+:
 
 3. 
 yes: a) b) c) 
 maybe not that d), as there is such a variety of regular expressions used
 
4. 
Non-working strategies:

a) Using the rewrite rules without changes.

From the reading: 

(10) 
(a)	k:v <=> u _ u C [#: | C]
(b)	k:[epsilon] <= V _ V C [#: | C]
 As u belongs to V (vowels), this will result in a conflict.

c) Subtracting the context of the more general rule from the more specific

It's nonsense, as the general rule will include the specific rule, conflict is inevitable, for example:

k:[epsilon] <=	[u - V]	_	[u - V] C [#: | C]
       			[u - V]	_	u C [#: | C]
       		      	u	_	[u - V] C [#: | C]


Working strategies:

b) Underspecifying the rewrite rules

(11) 
(a)	k:v <=> u _ u C [#: | C]	
(b)	k:[epsilon] | k:v <= V _ V C [#: | C]
Here (b) is underspecified, as there is | (or) in the left part, where there was a specified output in the rewrite rules.

d) Subtracting the context of the more specific rule from the more general.

(10) 
(b')	k:[epsilon] <=	[V - u]	_	[V - u] C [#: | C]
       				[V - u]	_	u C [#: | C]
       		      			u	_	[V - u] C [#: | C]

5.

 SOFT = ch sh tz _s _x
       [SOFT]<PL>:[SOFT]es 
       __<PL>:__s
0) I noticed but decided to disregard the fact that words where final consonants double are not covered by the given rule.
1) diagram: "drawing_2" in the folder
2) q5_plural.py in the folder




\end